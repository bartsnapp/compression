\documentclass{amsart}

\newcommand{\Z}{\mathbb{Z}}
\newcommand{\F}{\mathbb{F}}
\newcommand{\supp}{\mathop{\mathrm{Supp}}\nolimits}
\newcommand{\ep}{\varepsilon}

\begin{document}

\section*{Compression}


Basic idea: We want to encode data with polynomials.

So we need to be able to find polynomials with few coefficients.
\[
\supp(f) = \{\text{the number of nonzero coefficients of $f$}\}.
\]

\subsection*{One way to reduce coefficients}

There is a theorem that says:

Given any $\ep$, there exists $f$ such that
\[
\supp(f^2)\le \ep \supp(f)
\]

\subsection*{Another way}

Apply a transfromation (\textbf{probably} a linear shift) to find symmetry.

Qualities of the ``correct'' transformation:
\begin{enumerate}
  \item The equilvance classes of polynomials defined by
    \[
    f \sim g \Leftrightarrow f\circ T = g
    \]
    where $T$ is our transformation will have nice properties.
  \item It would be nice if there was some way to ``find'' the most
    symmetric transformation of the polynomial. I'm thinking calculus might help here.
\end{enumerate}

Take your favorite equivalence class, and then investigate which
transformations are needed for explicit polynomials to get the one
with the smallest support, and see if you can uncover a pattern for
chosing the correct transformation. Make a small table.


For example: Working in $\Z_{5}[x]$

\[
\begin{array}{|l|c|c|c|c|c|}\hline
  \text{polynomial} &
  x^3 &
  x^3+3 x^2+3 x+1 &
  x^3+x^2+2 x+3   &
  x^3+4 x^2+2 x+2 &
  x^3+2 x^2+3 x+4 \\ \hline
 \text{transformation} &  x   & x+1 & x+2 & x+3 & x+4 \\ \hline
\end{array}
\]

\subsection*{Work in a finite field of a power of two}

Work in $\F_{2^8}$.

Encode a circle.


\subsection*{Galois Theory}

Consider an equivalence class of polynomials by a (linear) transformation (start in $\Z_p[x]$).



\subsection*{Work in extension of a finite field}

So work in $\Z_p[\omega]$ where $\omega$ is algebraic over $\Z_p$





\end{document}
